\documentclass{article}
\title{Chess Editor 2.0}

\begin{document}
% Title Page
\begin{titlepage}
\begin{center}
\vspace{1cm}
\huge Chess Editor\\
\normalsize
\vspace{1cm}
\today\\
\vspace{1cm}
\begin{tabular}{c}
Devin M. O'Brien \\
Preston Williamson\\
Brandon Kyle\\
Dakota Simpkins\\
Tyler Wallschleger
\end{tabular}
\end{center}
\end{titlepage}
% Section 1 - Project Definition (Group Responsibility)
\section{Project Definition}
People like Chess and need a way to quickly develop and practice new strategies. I love
Chess!!! Chess Editor will let people play Chess on their PC and set up the board the
way they want to test different strategies, or just play in a different way. It will support
playing against an AI, or playing against another person locally. It will include an account
system to track wins and losses. For the purposes of testing, there will be an undo and
redo button, as well as tools to set up specific board scenarios. It will be programmed in
Java and we will use Stockfish for the AI.
\newpage
% Section 2 - Project Requirements (Group Responsibility)
\section{Project Requirements}

\subsection{Functional Requirments}

\subsubsection{Primary Requirements}
The primary requirements of the project must be functional at 'all times'. These requirements include the following:
\begin{enumerate}
\item User can play a game of chess:
\begin{itemize}
\item Locally (vs Player)
\item vs Computer
\end{itemize}

\item User can create, load, and play chess scenarios and variants like:


% Someone needs to attribute sources -_-
%% For each official scenario, add source under scenarios in the .bib
\begin{itemize}
\item Chess960 
\item 3-Check
\item King of The Hill
\cite{variants}
\item Horde % Needs Citation
\end{itemize}

\item The user is able to add a chess engine\footnote{Whether the chess engine works is dependent on the engine. However at least two public chess engines must work with the application.}.
\end{enumerate}
\subsubsection{Secondary Requirements}
The secondary requirements of the project may not always be function and are dependent on specific situation. These requirements include the following:
\begin{enumerate}
\item User can play a game of chess remotely.
\item User can use program to analyze moves.
\end{enumerate}
\subsection{Usability}
\subsubsection{User Interface}
The user interface must not be obtrusive and intuitive. Visuals need to be properly contrasted for differentiation. 
\subsubsection{Performance}
The program must be able to perform without noticable performance drop in all situations with the exceptional of analysis and secondary features.
\subsection{System}
\subsubsection{Hardware}
The required hardware is a standard personal computer with a monitor display.
\subsubsection{Software}
\paragraph{Required Software}
The required software includes:
\begin{itemize}
\item Graphviz: Reason
\end{itemize}
\paragraph{Operating System}
This program must be able to on both Windows 10 and Linux (specifically OpenSuse 15.2).

\subsubsection{Database}
(POSTPONED)
\subsubsection{Networking}
Certain secondary functionality requires a network connection in order to work properly.

\subsection{Security}
There are currently no security requirements with the exception of optional password protection for saved games and hosted games.
% Section 3  Project Specification (Group Responsibility)
\section{Project Specification}
\subsection{Scope}
\subsection{Libraries/ Frameworks/ Development Environments/ Dependencies}
\subsection{Platform}
\subsection{Genre}
% Section 4 Design Perspective (Group Responsibility)
\section{System - Design Perspective}
\subsection{Subsystems}
\subsubsection{Models}

% Section 5 Analyssis Perspective (Group Responsibility)
\section{System - Analysis Perspective}
\subsection{Subsystems}

% Section 6 Project Scrum Report (Group Responsibility)
\section{Project Scrum Report}

\subsection{Product Backlog} % Table / Diagram

\subsection{Sprint Backlog} % Table / Diagram

\subsection{Burndown Chart}

% Individual Responsibility
\section{Subsystems}
\subsection{Subsystem 1}
\subsubsection{Initial Design and Models}
\subsubsection{Data dictionary}
\subsubsection{Revisions}
\subsubsection{Scrum Backlog } % (Product and Sprint - Link to Section 6)
\subsubsection{Coding}
\paragraph{Approach}
\paragraph{Language}
\subsubsection{User Training}
\subsubsection{Testing}
\section{Complete System}

\section{References}

\bibliography{Project}
\bibliographystyle{ieeetr}
\end{document}
